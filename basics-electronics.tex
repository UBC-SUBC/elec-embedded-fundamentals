% This chapter is unfinished and not included, because there's really no way I could do this better than the many resources already out there.
\chapter{Basics of Electronic Circuits}

This chapter will give you a general overview of a simple model of electrical circuits. Many things happen in the real world that are not considered by this model, causing unintended behaviour. However, for simple circuits, this should be enough to understand how things work.
\section{Voltage, Current, and Resistance}
\subsection{Terminology}
Below is a general summary of some terms, along with their symbols in equations, if applicable.
\begin{itemize}
	\item Voltage (V): Electric potential. A good analogy, used by Professor Luis Linares (ECE), is that it is electric height. It will not cause anything to happen on its own unless given a path (i.e. a completed circuit) upon which to work. SI unit of volt (V).
	\item Current (I): Flow of electricity. A common analogy is water flow rate. SI unit of ampere (amp, A).
%	\item Power (P): Energy per unit time. In direct current (DC), this is defined as $P=VI$, and formulae for voltage and resistance, and current and resistance, can be derived via Ohm's law. In alternating current (AC), this is much more complicated. SI unit of watt (W).
	\item Resistance (r or R): the difficulty of passing an electric current through a conductor. Ohm's law, which states that $V=IR$, is the most important formula for resistance. From this, it can be inferred that increasing voltage or decreasing resistance will increase current. SI unit of Ohm ($\Omega$).
\end{itemize}

\subsection{Voltage Drops}
What does this mean in practice? The core concept here is that of a \textbf{voltage drop}. Voltage drops can be either positive or negative. Across a resistor