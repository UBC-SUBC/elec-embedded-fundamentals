\chapter{Introduction}
Welcome to SUBC! Whether you're in first or eighth year, we hope that this guide helps you navigate the world of electronics development for human-powered submarines. This guide is intended to be first an overview of some things that you should know, and second a reference material for when you just \emph{know} something, but it's evading you on the tip of your tongue.

As a primary onboarding measure, we intend to produce videos and run workshops covering much of the same content as is covered in this guide.

This set of resources (this guide, the videos, and the workshops) are \emph{not} meant to serve as an in-depth exploration of theory; theory is covered if it is essential to understanding how to do something in practice. For theory, or if you'd prefer a different tone, there are many great resources available, including, but not limited to: classes, textbooks, YouTube, and MIT OpenCourseWare. A list of some things that we have found helpful is in the appendix.

\section{Safety Note}
Please make sure to follow all safety guidelines outlined in this guide and given by the team, workspace, or university. In general, here is a by-no-means comprehensive set of rules of thumb to follow:
\begin{itemize}
	\item High voltages are often dangerous.
	\item Anything with sharp or hot things should only be done with safety glasses.
	\item If you don't know how to do, or have never done, something, ask for help before doing it.
	\item Don't download anything from the internet that seems suspicious.
\end{itemize}
\emph{Remember, your safety is the number-one priority. A safe team is a happy and productive team.}

\section{Correctness Disclaimer}
This guide has been written by undergraduate engineering students. Therefore, it is likely to contain many errors, moreso than many other resources available. If you find an error, \emph{please} report it to us so that it can be fixed as soon as possible. This can be done by sending us an email at \href{mailto:hello@subc.ca}{hello@subc.ca}, opening an issue on the \href{https://github.com/UBC-SUBC/elec-embedded-fundamentals}{GitHub repository}, or fixing the change through a pull-request.

\section{Thanks}
Many thanks to Vincent Yuan, who created the first iteration of SUBC electronics workshops.
